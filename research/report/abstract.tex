\begin{abstract}
  Efficiently routing vehicle traffic is fast becoming a key area of research. With the number of vehicles on the road increasing, existing infrastructure is under growing strain. Extending road networks is a costly process and requires ongoing, expensive maintenance.

  Human drivers do not make efficient use of the existing road network, and so many approaches are being investigated for increasing this efficiency, thereby increasing throughput on the same series of roads.

  In this paper I investigate the potential applications of Genetic Algorithms in this space. Building and evaluating a parallel cooperative coevolutionary algorithm (PCCGA), utilising Bézier curves to encode individuals in each sub-population.

  I show that this approach is suitable for parallelisation and deploy it across 16 threads.

  I conclude that this approach has the potential, given more research and development time, to consistently produce feasible, safe routes for multiple agents across a network of roads.


\end{abstract}
%%% Local Variables:
%%% mode: latex
%%% TeX-master: "report"
%%% End:
