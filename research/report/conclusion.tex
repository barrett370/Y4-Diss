
This is clearly a very broad and difficult area of research. I have shown that the task of safely routing many vehicles through interconnected sections of road can be thought of as multiple, large, sub-problems, many of which are seeing highly targeted research.

I think I have shown the merits of my approach: the generative nature of GAs as well as their potential for massive parallelisation along with the smooth, continuous and malleable properties of Bézier curves have been shown to produce good quality, feasible routes for multiple agents.

However, it is also clear that substantially more research is still required to strip away the levels of abstraction and remove the naive assumptions before any such system could realistically see real-world deployment. As mentioned previously \todo{did I?}, any system seeking real-world use would require substantial government backing, likely requiring a mandated standard protocol and level of autonomous capability to be present in all new cars for a large amount of time, such that the vast majority of road-legal vehicles adhear to it, otherwise the system would need to be extended to plan around human-guided vehicles.

In my opinion, the assumption that all vehicles are fully-autonomous is not so much an assumption as it is a functional requirement for any system such as this to be efficient and achieve the net reduction in travel time that it attempts to.

%TC:macro \todo 1

%%% Local Variables:
%%% mode: latex
%%% TeX-master: "report"
%%% End:
