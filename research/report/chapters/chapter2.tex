\section{Genetic Algorithms}

Genetic algorithms are optimisation techniques that employ the same rationale as classical Evolution as seen in nature.

In a general sense, optimisation techniques work to find the set of parameters $\mathcal{P}$ that minimise an objective function $\mathcal{F}$. 
Genetic algorithms approach this by representing these sets as individuals in a population, $P$. Over the course of multiple generations, the best solutions are determined and promoted until termination criteria are met or the maximum number of generations is reached.

As our candidates are essentially a collection of parameters to the function we are trying to optimise, we can extend our metaphor further by mapping each element of a individual to a \textit{gene} in a individual's genome. 

The representation we use in a GA is problem specific. Often we have to provide functions to facilitate the mapping between the problem specific set of possible solutions and the encoded genotype space in which we optimise.

Genetic algorithms are both \textit{probabilistically optimal} and \textit{probabilistically complete}\cite{kalaOnroadIntelligentVehicles2016} meaning that given infinite time, the algorithm \textbf{will} find a solution, if one exists, and it \textbf{will} be the \textit{most} optimal.

\begin{algorithm}[H]
	\label{GenericGA}
	\SetAlgoLined
	\KwResult{Best Solution, $p_{ \texttt{best}}$}	
	Generate initial population, $P_0$ of size $n$\;
	Evaluate fitness of each individual in $P_0$, $\{F(p_{0,1},\ldots, p_{0,n})\}$\;
	\While{termination criteria are not met}{
		\textbf{Selection}: Select individuals from $P_t$ based on their fitness\;
		\textbf{Variation}: Apply variation operators to parents from $P_t$ to produce offspring\;
		\textbf{Evaluation}: Evaluate the fitness of the newly bred individuals\;
		\textbf{Reproduction}: Generate a new population $P_{t+1}$ using individuals from $P_t$ as well as the newly bred candidates.\;
		$t$++
	}
	return $p_{\texttt{best}}$

	\caption{Generic Genetic Algorithm}
\end{algorithm}


\section{Autonomous Road Networks}

\section{Quantum Genetic Algorithms}
