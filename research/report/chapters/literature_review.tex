
\section{Classical GAs}

Classical Genetic algorithms have seen research and applications in a number of fields over the past 40 years ranging from Heart disease diagnosis\cite{reddyHybridGeneticAlgorithm2020} to predicting the strength of concrete under various conditions\cite{shariatiPredictionConcreteStrength2020}.

There is a substantial and growing body of research specifically looking at the applications of GAs in route planning for autonomous agents, however, many are focusing on either abstract robotic agent planning or \todo{wording?} military focused applications such as autonomous drones\cite{robergeFastGeneticAlgorithm2018}.


\section{Bézier curves for route generation}

\textbf{See annotated copy of Quartic Bézier curve based trajectory generation for autonomous vehicles with curvature and velocity constraints \cite{chenQuarticBezierCurve2014}}\todo{Expand this}

\section{Fully Autonomous Road Networks}

\textit{FARN}s have the possibility of improving travel in a number of different areas. 

They have the potential of greatly reducing travel times by efficiently routing all vehicles with the goal of reducing the net travel time. Such a system would also be able to respond much faster than human drivers, allowing for large increases in permissible vehicle velocities thus, further increasing efficiency.

By extension of their higher efficiency, \textit{FARN}s will also have a marked effect on vehicular energy consumption. The potential effects are summarised nicely by Vahidi and Sciarretta\cite{vahidiEnergySavingPotentials2018} where they show the use of vehicle automation could cause anywhere from a halving of "energy use and greenhouse gas emission" in an "optimistic scenario" to doubling them depending on the effects that present themselves. The increase in highway speed whilst increasing travel efficiency is predicted by Brown et al.\cite{brownAnalysisPossibleEnergy2014} and Wadud et al.\cite{wadudHelpHindranceTravel2016} as increasing energy use by anywhere from 5\% to 30\%

