\message{ !name(../report.tex)}\documentclass[a4paper,11pt]{report}
\usepackage[T2A]{fontenc}
\usepackage[utf8]{inputenc}
\usepackage[main=english,russian]{babel}
\usepackage{graphicx}
\usepackage{csquotes}
\usepackage[toc,page]{appendix}
\usepackage[ruled,vlined]{algorithm2e}
\setlength {\marginparwidth}{2cm}
\usepackage{todonotes}
\graphicspath{ {images/} }
\hfuzz 6pt
\usepackage{hyperref}
\usepackage{tabularx,lipsum,environ,amsmath,amssymb,framed}

\makeatletter
\newcounter{problemCounter}
\newcommand{\problemtitle}[1]{\gdef\@problemtitle{#1}}% Store problem title
\newcommand{\probleminput}[1]{\gdef\@probleminput{#1}}% Store problem input
\newcommand{\problemquestion}[1]{\gdef\@problemquestion{#1}}% Store problem question
\newcommand{\theProblemCounter}{\arabic{problemCounter}}
\NewEnviron{problem}[2]{
    \begin{framed}
  \problemtitle{}\probleminput{}\problemquestion{}% Default input is empty
  \BODY% Parse input
  \par\addvspace{.5\baselineskip}
  \noindent
  \begin{tabularx}{\textwidth}{@{\hspace{\parindent}} l X c}
    \multicolumn{2}{@{\hspace{\parindent}}l}{\@problemtitle} \\% Title
    \textbf{Input:} & \@probleminput \\% Input
    \textbf{Question:} & \@problemquestion% Question
  \end{tabularx}
  \par\addvspace{.5\baselineskip}
  \end{framed}
  \refstepcounter{problemCounter}%
  \label{#1}
  \noindent\textbf{Problem~\theProblemCounter}
}
\makeatother



\title{
	{Applications of Genetic Algorithms on theoretical fully-autonomous road systems} \\
	{\large University of Birmingham} \\ 
	{\includegraphics[scale=0.3]{uobcrest.jpg}}
}
\author{Sam Barrett, \ID \\ sjb786@student.bham.ac.uk \\ Programme: \programme \\ Supervisor: \supervisor \\ Word count: \input{|./wordcount.sh}}
\newcommand{\ID}{1803086}
\newcommand{\supervisor}{Miriam Backens}
\newcommand{\programme}{MSci Computer Science}



\begin{document}

\message{ !name(chapters/literature_review.tex) !offset(-57) }


\section{Genetic algorithms for route generation}


Classical Genetic algorithms have seen research and applications in a number of fields over the past 40 years ranging from Heart disease diagnosis\cite{reddyHybridGeneticAlgorithm2020} to predicting the strength of concrete under various conditions\cite{shariatiPredictionConcreteStrength2020}.

There is a substantial and growing body of research specifically looking at the applications of GAs in route planning for autonomous agents, however, many are focusing on either abstract robotic agent planning in discrete search spaces or failing to take into account the wider environment of multiple agents.
\todo[inline]{Kala's book \& papers}

\subsection{Genetic algorithms for cooperative route generation}

\paragraph{Raul Kala - Optimization Based Planning 2016\cite{kalaOptimizationBasedPlanning2016}}
In his book, Kala proposes a GA as an embedded component of a larger planning system which also relies on Djiksra's algorithm for macro-level route planning along with the classical autonomous vehicle real-time collision avoidance toolbox (radarr, sonarr, etc.). In order to achieve a system capable of planning for multiple agents concurrently, Kala proposed the use of these real-time collision avoidance tools along with embedding \textit{traffic rules} as heuristics into the Djiksta-based component. These traffic rules included ``driving on the left'' and ``overtaking on the right'' etc. This proposed system, while similar to mine, differs in scope. I propose a system for planning routes for a set of autonomous agents within a fully-autonomous road system, meaning agents will not (in theory) interact or encounter any other vehicles which are not a part of the planning network.

\paragraph{Luis Cruz-Piris et al. - Automated Optimsization of Intersections Using a Genetic Algorithm\cite{cruz-pirisAutomatedOptimizationIntersections2019}}
A paper published by Cruz-Piris et al. investigates the applications of Genetic Algorithms on routing traffic through busy intersections. Their proposed system is shown to have improvements in traffic throughput of anywhere from 9 to 36\%.

Their approach involved several assumptions, many of which my research shares. Namely, they assumed vehicles do not stop at any point and that their speed remains constant at all points in time. They modelled intersections and associated traffic flows using (Traffic) Cellular Automata as proposed by Maerivoet and De Moor\cite{maerivoetCellularAutomataModels2005}



\section{Bézier curves for route generation}

\textbf{See annotated copy of Quartic Bézier curve based trajectory generation for autonomous vehicles with curvature and velocity constraints \cite{chenQuarticBezierCurve2014}}\todo{Expand this}

\section{Fully Autonomous Road Networks}

\textit{FARN}s have the possibility of improving travel in a number of different areas. 

They have the potential of greatly reducing travel times by efficiently routing all vehicles with the goal of reducing the net travel time. Such a system would also be able to respond much faster than human drivers, allowing for large increases in permissible vehicle velocities thus, further increasing efficiency.

By extension of their higher efficiency, \textit{FARN}s will also have a marked effect on vehicular energy consumption. The potential effects are summarised nicely by Vahidi and Sciarretta\cite{vahidiEnergySavingPotentials2018} where they show the use of vehicle automation could cause anywhere from a halving of ``energy use and greenhouse gas emission'' in an ``optimistic scenario'' to doubling them depending on the effects that present themselves. The increase in highway speed whilst increasing travel efficiency is predicted by Brown et al.\cite{brownAnalysisPossibleEnergy2014} and Wadud et al.\cite{wadudHelpHindranceTravel2016} as increasing energy use by anywhere from 5\% to 30\%


%%% Local Variables:
%%% TeX-master: "../report"
%%% End:

\message{ !name(../report.tex) !offset(-21) }

\end{document}



%%% Local Variables:
%%% TeX-master: t
%%% End:
