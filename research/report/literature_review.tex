

\section{Genetic algorithms for route generation}


Classical Genetic algorithms have seen research and applications in a number of fields over the past 40 years ranging from Heart disease diagnosis\cite{reddyHybridGeneticAlgorithm2020} to predicting the strength of concrete under various conditions\cite{shariatiPredictionConcreteStrength2020}.

There is a substantial and growing body of research specifically looking at the applications of GAs in route planning for autonomous agents, however, many are focusing on either abstract robotic agent planning in discrete search spaces or failing to take into account the wider environment of multiple agents.

\subsection{Genetic algorithms for cooperative route generation}
\label{subsec:lit_rev-GACoopRoutes}


\paragraph{Cai \& Peng - Cooperative Coevolutionary Adaptive Genetic Algorithm in Path Planning of Cooperative Multi-Mobile Robot Systems\cite{caiCooperativeCoevolutionaryAdaptive2002}}

In this 2001 paper, Cai \& Peng give an example of a Cooperative Coevolutionary Adaptive Genetic Algorithm (CCAGA) for planning routes for multiple autonomous agents.

Their approach features a discrete grid-based search space featuring rectangular obstacles which the routes must avoid. Their chromosomes are encoded using real-values representing the $x$ and $y$ coordinates of the search grid.

Their fitness function has two stages, in the first generation the fitness of a candidate solution is purely determined by the global \textit{optimality} of the route. In all subsequent generations the fitness also takes into account the best routes from each of the other sub-populations.

They conclude that their approach yields good robustness and convergence as well as being suitable for parallel execution, allowing for systems employing this GA approach to run very efficiently.


\paragraph{Elshamli et al. - Genetic algorithm for dynamic path planning}\cite{elshamliGeneticAlgorithmDynamic2004}

In this 2004 paper Elshamli et al. propose a Genetic Algorithm Planner (GAP) which generates routes made up from edges connecting $n$ points. They penalise routes for colliding with obstacles as well as for not being \textit{smooth}, the core of their route fitness is based on route length.

The smoothness metric is based on the angles between the edges as they connect at a node.

They do not provide concrete evaluations for their implementation but seem to focus on the notion of \textit{dynamic} search spaces and which of their approaches handle these best. They define their dynamic search spaces as spaces in which obstacles are added during generation, for example, after the first generation. Their proposed solutions to this are as follows:

\begin{itemize}
  \item Utilise high mutation rate of 50\% and a low crossover rate of 50\%
  \item Introduce newly randomly generated individuals at the beginning of each generation
  \item A method by which to remember the fittest individuals at the end of each generation and to reintroduce them with a certain chance
  \item A combination of the previous two
\end{itemize}

In their testing the best average solution was achieved using the first of these approaches and the best overall solutions were achieved using the final solution.

All of these approaches could potentially be utilised in a cooperative, multi-agent, approach if we consider other agents to be dynamic obstacles.

\paragraph{Raul Kala - On-Road Intelligent Vehicles: Optimization Based Planning 2016\cite{kalaOptimizationBasedPlanning2016}}
In his book, Kala proposes a GA as an embedded component of a larger planning system which also relies on Djiksra's algorithm for macro-level route planning along with the classical autonomous vehicle real-time collision avoidance toolbox (radar, sonar, etc.). In order to achieve a system capable of planning for multiple agents concurrently, Kala proposed the use of these real-time collision avoidance tools along with embedding \textit{traffic rules} as heuristics into the Djiksta-based component. These traffic rules included ``driving on the left'' and ``overtaking on the right'' etc. This proposed system, while similar to mine, differs in scope. I propose a system for planning routes for a set of autonomous agents within a fully-autonomous road system, meaning agents will not (in theory) interact or encounter any other vehicles which are not a part of the planning network.

Kala also proposes a seemingly novel \textit{road space coordinate system} which allows him to formally bound the genotypic search space by the dimensions of the road, eliminating the possibility of plotting routes outside of the road space. He goes on to show that this approach in his implementation greatly reduces the number of agents required to generate feasible solutions, though it is not discussed how this may impair the GAs ability to generate complex routes which may require control points to lie outside of the road space whilst still defining a feasible curve.

\paragraph{Cruz-Piris et al. - Automated Optimsization of Intersections Using a Genetic Algorithm\cite{cruz-pirisAutomatedOptimizationIntersections2019}}

A paper published by Cruz-Piris et al. investigates the applications of (binary coded) Genetic Algorithms on routing traffic through busy intersections. Their proposed system is shown to have improvements in traffic throughput of anywhere from 9\% to 36\%.

Their approach involved several assumptions, many of which my research shares. Namely, they assumed vehicles do not stop at any point and that their speed remains constant at all points in time.

Their approach to modelling the solution space takes advantage of the fairly uniform and regular structure of large intersections, and although they work on modelling irregularly shaped intersections, this traffic cellular automata (TCA) model begins to fall apart when the size of each cell cannot contain a vehicle and requisite surrounding space, or if input/output lanes do not lie in a regular grid pattern. Facilitating irregularly shaped intersections seems to be a manual process, not a particularly viable method when you consider the number of unique intersections across the world or even just the Continental United States!

This approach seems to work relatively well for throughput optimisation problems such as junctions but does not scale well for less structured scenarios where all origins and goals cannot be known beforehand and are not necessarily the same for all subsequent runs.

\paragraph{Roberge et al. - Fast Genetic Algorithm Path Planner for Fixed-Wing Military UAV Using GPU\cite{robergeFastGeneticAlgorithm2018}}

In this 2018 paper an interesting proposition was made, greatly increasing efficiency and viability of real-time planning via GAs through massively parallelising its core processes to run on GPUs. Their implementation shows a 290x times speedup compared with conventional sequential CPU execution. GAs lend themselves quite nicely to parallel execution as we are applying the same operators to multiple individuals of the same shape.

\section{Bézier curves for route generation}

\paragraph{Chen et al. - Quartic Bézier Curve based Trajectory Generation for Autonomous Vehicles with Curvature and Velocity Constraints}

In this 2014 paper, Chen et. al research the efficacy of quartic (4-degree) Bézier curves for planning continuous routes between points for autonomous vehicles. In order to find an optimal solution they employ sequential quadratic programming.

In their approach Chen et al. also take into account the orientation, change in orientation and velocity of the routes generated in order to assure all generated routes are \textit{safe} for occupied vehicles to follow.

This approach appears to be extremely computationally efficient on the toy examples provided in the paper, on relatively low performance hardware 560 iterations can be computed in around 500ms!

There is no mention, however, to a cooperative element. Explicitly solving quadratic programming tasks is known to be $\mathbf{NP}$-complete as shown by Vavasis in 1990\cite{VAVASIS199073} and adding onto that the requirement that all solutions interact nicely will only increase the real-world runtime of any solution. A real-world scenario with hundreds of vehicles may suffer from an incredibly high runtime, hardware requirement or both.

\section{Fully Autonomous Road Networks}

Fully Autonomous Road Networks (\textit{FARN}s) have the possibility of improving travel in a number of different areas.

They have the potential of greatly reducing travel times by efficiently routing all vehicles with the goal of reducing the net travel time. Such a system would also be able to respond much faster than human drivers, allowing for large increases in permissible vehicle velocities thus, further increasing efficiency.

By extension of their higher efficiency, \textit{FARN}s will also have a marked effect on vehicular energy consumption. The potential effects are summarised nicely by Vahidi and Sciarretta\cite{vahidiEnergySavingPotentials2018} where they show the use of vehicle automation could cause anywhere from a halving of ``energy use and greenhouse gas emission'' in an ``optimistic scenario'' to doubling them depending on the effects that present themselves such as reduced public transport usage. The increase in highway speed whilst increasing travel efficiency is predicted by Brown et al.\cite{brownAnalysisPossibleEnergy2014} and Wadud et al.\cite{wadudHelpHindranceTravel2016} as increasing energy use by anywhere from 5\% to 30\%.


%TC:macro \todo 1
%%% Local Variables:
%%% TeX-master: "../report"
%%% End:
