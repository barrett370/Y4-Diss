
\todo[inline]{Lay out different tasks \& steps of the project, identify subgoals}
\todo[inline]{Move formal goal definitions to Evaluation? }
\section{Goals}

The overarching goal of this project is to optimally route traffic in the setting of a fully-autonomous road system. This however, is an aspirational goal with many sub-requirements to be fulfilled before it can come to fruition.

Formally this top-level goal can be said to be to answer the question in Problem~\ref{prob:Spec}.

\begin{figure}[htpb]
    \centering
    \begin{problem}{prob:Spec}

        \problemtitle{Top-level Goal}
        \probleminput{ 
            \begin{itemize}
                \item A road system $\mathcal{R}$ represented as a graph $\mathcal{R} = (V,E)$


                    Where each edge $e \in E$ is a section of road defined as the space between two vertices $V^1,V^2 \in V$ which is bounded by two functions $b_1(x)$ and $b_2(x)$ and augmented by a set of obstacles $O$ representing infeasible sections of road space
                \item a set of agents, $A = \left\{ a_1,\ldots,a_n \right\}$
                \item a set of vertex pairs representing start and goal points for each agent: 

                    $\mathcal{V} = \left\{ (V_1^1,V_1^2),\ldots, (V_n^1, V_n^2) \right\}, V_{i \in [1 \ldots n]}^{j \in [1,2]} \in V$ 

                where the route for agent $a_i$ is from vertex $V_i^1$ to $V_i^2$
            \end{itemize}
        }
        \problemquestion{ What is the set of optimal routes $\mathcal{V}_{optimal}$, where the $i^{\text{th}}$ element  is defined as a set of linked Bézier curves connecting $V_i^1$ and $V_i^2$ through feasible space?}
        \end{problem}
\end{figure}

We can then split this into more manageable sub-goals. The sub-problem of generating a route through a section of road is defined in Problem~\ref{prob:sub1}. This can be further decomposed into the selfish routing of a single agent through a section of road defined in Problem~\ref{prob:sub2}

\begin{figure}[htpb]
    \centering
    \begin{problem}{prob:sub1}
        \problemtitle{Sub-Goal: Cooperative route planning}
        \probleminput{
            \begin{itemize}
                \item A start point $P_{start}$ and a goal point $P_{goal}$ 
                \item A section of road as defined in Problem~\ref{prob:Spec}
                \item A knowledge of all routes being planned to be executed concurrently.
            \end{itemize}
        }
        \problemquestion{What is the optimal route, in the form of a Bézier curve, between these two points s.t. it does not collide with any other agents or pass through any infeasible regions in the road space?}
    \end{problem}
\end{figure} 

\begin{figure}[htpb]
    \centering
    \begin{problem}{prob:sub2}
        \problemtitle{Sub-Goal: Single agent route planning}
        \probleminput{
            \begin{itemize}
                \item A start point $P_{start}$ and a goal point $P_{goal}$ 
                \item A section of road as defined in Problem~\ref{prob:Spec}
            \end{itemize}
        }
        \problemquestion{What is the optimal route, in the form of a Bézier curve, between these two points s.t. it does not pass through any infeasible regions in the road space?}
    \end{problem}
\end{figure} 

\begin{figure}[htpb]
    \centering
    \begin{problem}{prob:sub3}
        \problemtitle{Sub-Goal: Bézier curve generation}
        \probleminput{
            \begin{itemize}
                \item A start point $P_{start}$ and a goal point $P_{goal}$ 
                \item A section of road as defined in Problem~\ref{prob:Spec}
            \end{itemize}
        }
        \problemquestion{What is the optimal route, in the form of a Bézier curve, between these two points s.t. it does not pass through any infeasible space?}
    \end{problem}
  \end{figure}

\subsection{Requirements}
\label{subsec:requirements}

The goal of this project was not to produce a production-ready system, but instead, to investigate the plausibility of GAs on the real future possibility of completely autonomous road networks. As such I feel it useful to outline the theoretical requirements of a production grade system. To do this formally I will employ Propositional and Temporal logic.

\begin{enumerate}
\item The system should never return a set of routes such that, for any time $t \in T, \forall i \in n, \forall j \in n, j \neq n, I_{i}(t) = I_{j}(t)$. I.e. for any time, no routes should inhabit the same point, meaning there are no collisions in the planned routes.\todo{This is wrong, correct this or remove it outright}
\end{enumerate}

%TC:macro \todo 1
%%% Local Variables:
%%% TeX-master: "../report"
%%% End:
