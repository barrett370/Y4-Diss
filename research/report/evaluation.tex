\subsection{Genetic Algorithms}

The core of my approach to solving the problems outlined in Chapter~\ref{chap:Intro} was creating a Genetic algorithm to evolve populations of candidate routes, returning the fittest once the termination criteria are met.

\subsection{Bézier Curves}
\todo[inline]{expand on this section, talk about issues of finding intersection, possible GPU applications}

Bézier curves have been utilised in this project to encode and represent the route of a vehicle. As mentioned in Section~\ref{sec:back-bezier-curves}, there are many reasons I originally selected them for this task. However, over the course of implementation and testing, a number of downsides have been presented.

\begin{enumerate}
  \item Objective\todo{correct word?} numerical approaches with Bézier curves are often complicated, expensive and do not generalise well to $n$ control points.

        This leads to many heuristics, and approximations being employed to save computation. Approximations and assumptions in a system as the one proposed in this report, are sub-optimal and could potentially lead of undesired behaviour which could ultimately have dire consequences if such a system were to be deployed.

        Other research such as that by Cai \& Peng\cite{caiCooperativeCoevolutionaryAdaptive2002} takes a different approach, using discrete, grid-based search spaces in which routes are made up of a series of connected straight line segments. This approach removes much of the complexity from my solution but introduces its own concerns.

        The routes generated by Cai \& Peng's approach are intended to be executed by autonomous robots, so no thought has been given to potential passengers. Consequently, these routes would require smoothing as a post-planning process, re-introducing complexity.

        Another possible representation is the approach taken by Cruz-Piris et al.\cite{cruz-pirisAutomatedOptimizationIntersections2019} which involved representing the section of road, in their case an intersection, as a cellular automata in which a single vehicle can occupy a single cell at any given point in time. \feedback{is this repeating what I said in the literature review too much?}
\end{enumerate}\todo{does this need to be an enumerated list?}


There were however, also some advantages and nice properties of Bézier curves which lent themselves to the task.

Their relatively simple abstract construction as a series of control points proved easy to represent as a genotype. This made the creation of the various genetic operators relatively simple, requiring little pre or post-processing.

They are also capable of representing a high complexity of curve in a relatively simple and concise manner. This makes code much more approachable and algorithms easier to digest.

\subsection{Cooperative Planning}
\label{subsec:eval-cooperativeplanning}

My solution to the problem of planning $n$ non-colliding routes for $n$ agents was so wrap my existing \texttt{GA} function in a cooperative \textit{layer}. This cooperative layer relied on a function for detecting collisions which had extremely high overhead, at one point causing around 50x slowdown in the running time of the function. Detecting intersections between two Bézier curves is a non-trivial task with the best methods taking the same approach of recursive subdivision that I utilised.

\todo[inline]{make mention of possible GPU implementations such at in~\cite{robergeFastGeneticAlgorithm2018}, modern enterprise GPUs have around 4000-10000 cores, approximately the max number of curve splits and comparisons in a binary check of depth 6. Being able to do this in a single cycle would result in a huge speedup, paper uses Titan X }



%TC:macro \todo 1

%%% Local Variables:
%%% mode: latex
%%% TeX-master: "report"
%%% End:
