\section{Results}

Due to the high dimensionality of my search space, I cannot visualise the objective function of my GA.

My results were achieved by
\subsection{Bézier Curves}
\todo[inline]{expand on this section, talk about issues of finding intersection, possible GPU applications}

Bézier curves have been utlised in this project to encode and represent the route of a vehicle. As mentioned in Section~\ref{sec:back-bezier-curves}, there are many reasons I originally selected them for this task. However, over the course of implementation and testing, a number of downsides have been presented.

\subsection{Cooperative Planning}
\label{subsec:eval-cooperativeplanning}

My solution to the problem of planning $n$ non-colliding routes for $n$ agents was so wrap my existing \texttt{GA} function in a cooperative \textit{layer}. This cooperative layer relied on a function for detecting collisions which had extremely high overhead, at one point causing around 50x slowdown in the running time of the function. Detecting intersections between two Bézier curves is a non-trivial task with the best methods taking the same approach of recursive subdivision that I utilised.

\todo[inline]{make mention of possible GPU implementations such at in~\cite{robergeFastGeneticAlgorithm2018}, modern enterprise GPUs have around 4000-10000 cores, approximately the max number of curve splits and comparisons in a binary check of depth 6. Being able to do this in a single cycle would result in a huge speedup }



%TC:macro \todo 1
%%% Local Variables:
%%% TeX-master: "../report"
%%% End:
